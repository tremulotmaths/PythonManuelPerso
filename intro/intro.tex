%%%%%%%%%%%%%%%%%%%%%%%%%%%%%%%%%%%%%%%%%%%%%%%%%%%%%%%%%%%%%%%%
\section{Qu'est-ce qu'un algorithme ?}
%%%%%%%%%%%%%%%%%%%%%%%%%%%%%%%%%%%%%%%%%%%%%%%%%%%%%%%%%%%%%%%%

\begin{Defi}[]{}
Un algorithme est une liste finie de processus élémentaires, appelés instructions élémentaires, amenant à la
résolution d'un problème.
\end{Defi}

\begin{Exemple}[s]{}
$\star$ Une recette de cuisine

$\star$ Un trajet élaboré par un GPS

$\star$ Le plan de montage d'un meuble
\end{Exemple}

%%%%%%%%%%%%%%%%%%%%%%%%%%%%%%%%%%%%%%%%%%%%%%%%%%%%%%%%%%%%%%%%
\section{Les langages}
%%%%%%%%%%%%%%%%%%%%%%%%%%%%%%%%%%%%%%%%%%%%%%%%%%%%%%%%%%%%%%%%

\begin{CadreAlgo}{\linewidth}{}
Un algorithme peut être décrit en langage \og naturel\fg{}, mais on utilise dans la plupart des cas un langage plus
précis adapté aux instructions utilisées : on parle alors de langage de programmation.
\end{CadreAlgo}

Cette année, nous utiliserons principalement\dots

$\star$ le langage \og naturel \fg{},

$\star$ le langage de programmation de votre calculatrice,

$\star$ et le langage de programmation Python.

\medskip

De façon générale, on peut considérer trois étapes dans un algorithme :

\begin{enumerate}
\item \textbf{L'entrée des données}

Dans cette étape figure la lecture des données qui seront traitées au cours de l'algorithme. Ces données peuvent
être saisies au clavier ou bien être lues dans un fichier annexe.

\item \textbf{Le traitement des données}

C'est le cœur du programme. Il est constitué d'une suite d'instructions, parmi lesquelles les différentes opérations
sur les données, qui permettent de résoudre le problème.

\item \textbf{La sortie des résultats}

C'est le résultat obtenu qui peut être affiché à l'écran ou enregistré dans un fichier.

\end{enumerate}